\normalsize
\section{Incompressibility Constraint}

The Navier's equation \cite{Wilcox2003} can be derived from the conservation of momentum:
\begin{equation}
\rho \f{D \+u}{D t}=\di \+{\+\sigma} + \rho \+f
\end{equation}
where \\
$\rho$ is the density, \\
$\+u $ is the velocity vector, \\
$\+f $ is the body force vector, and \\
$\+{\+\sigma} $ is the stress tensor that can be decomposed into the viscous stress tensor $\+{\+\tau} $ and the pressure, \\
\begin{equation}
\+{\+\sigma} = -p \+{\+\delta} + \+{\+\tau}
\end{equation}
where $\+{\+\delta}$ is the identity tensor.

To develop the constitutive relationship for Newtonian fluid, the Stokes' postulate \cite{Stokes1845,Batchelor1967,Panton1996,White1991} contains the following ideas: (1) The viscous stress tensor is a symmetric tensor; (2) there is no viscous stress if the rate of strain is zero; (3) the viscous stress per strain rate ratio is isotropic.

With these assumptions the stress tensor can be derived for the three-dimensional Cartesian coordinate:
\be
\sigma_{ij} = -p\delta_{ij} + 2\mu S_{ij}+\delta_{ij} \lambda \f{\p u_k}{\p x_k} %\n \cdot u
\label{eqn:sigma-stress}
\ee
where $\lambda$ is the second viscosity and $S_{ij}$ are the strain tensor components,
\be
S_{ij}=\f{1}{2}(\f{\p u_i}{\p x_j}+\f{\p u_j}{\p x_i})
\ee
The Stokes' hypothesis relates the two viscosity coefficients,
\be
\lambda + \f{2}{3}\mu = 0
\ee
therefore the coefficient $\lambda$ can be eliminated from Equation \ref{eqn:sigma-stress},
\be
\sigma_{ij} = -p\delta_{ij} + \mu(\f{\p u_i}{\p x_j}+\f{\p u_j}{\p x_i}-\delta_{ij}\f{2}{3}  \f{\p u_k}{\p x_k}) %\n \cdot u
\ee
and the momentum equation can be written as:
\begin{equation}
\rho \f{D u_i}{D t}=
-\f{\p p}{\p x_i} +\f{\p}{\p x_j}[\mu (\f{\p u_i}{\p x_j}+\f{\p u_j}{\p x_i}-\delta_{ij}\f{2}{3}  \f{\p u_k}{\p x_k})]
+ \rho f_i
\end{equation}
\begin{comment}
\begin{equation}
\rho (\frac{\p u_i}{\p t}+\frac{\p u_i u_j}{\p x_j})=-\f{\p p}{\p x_i} +\mu\f{\p^2 u_i}{\p x_j \p x_j} + \f{2}{3} \delta_{ij}\f{\p}{\p x_j}(\f{\p u_k}{\p x_k})  + \rho f_i
\end{equation}
\begin{equation}
\rho (\frac{\p u_i}{\p t}+u_j \frac{\p u_i}{\p x_j}+u_i \f{\p u_k}{\p x_k})=-\f{\p p}{\p x_i} +\mu\f{\p^2 u_i}{\p x_j \p x_j} + \f{2}{3} \delta_{ij}\f{\p}{\p x_j}(\f{\p u_k}{\p x_k})  + \rho f_i
\label{eqn:incompressible}
\end{equation}
\end{comment}
The term {\Large $\f{\p u_k}{\p x_k}$} is the velocity divergence and vanishes for incompressible flows. However, liquid water is not perfectly incompressible but slightly compressible. To quantify the compressibility, the bulk modulus is defined as:
\be
B = \rho \f{\p p}{\p \rho}
\ee
The value for liquid water is $2.2 \ GPa$ at $25^o C$. This means that $0.05 \%$ change of density requires $1 \  MPa$ of pressure change, or about ten times of the atmospheric pressure. ($1Pa = 1 N/m^2 = 1 kg/m \cd s^2$, $1 Atm = 101,325 Pa$ ). For the numerical implementation, if the liquid water governing equation is computed without the assumption of incompressibility, $0.05 \%$ divergence error will result in $1 MPa$ of unreal pressure oscillation. Besides, the pressure wave in liquid water propagates at the speed of around $1480 m/s$ at $20^o C$, requiring extremely small time step to resolve the correct pressure and the fluid velocities. Although the presence of air bubbles may decrease the speed of pressure wave significantly.
\be
c=\sqrt{B/\rho}=\sqrt{\f{2.2 GPa}{1000 kg/m^3}}=1483 m/s
\ee
If the assumption of incompressibility is made, then the momentum equation can be simplified as:
\begin{equation}
\rho \f{D u_i}{D t}=
-\f{\p p}{\p x_i} +\f{\p}{\p x_j}[\mu (\f{\p u_i}{\p x_j}+\f{\p u_j}{\p x_i})]
+ \rho f_i
\label{eqn:incompressible}
\end{equation}
\begin{comment}
It can as well be written in the vector form,
\begin{equation}
\rho (\frac{\p u}{\p t}+u \n u)=-\n p +\mu \n^2 u + \rho f
\label{eqn:incompressible-vector}
\end{equation}
\end{comment}
When numerically integrating this Navier-Stokes equation, the incompressibility constraint is not satisfied automatically. To meet this criteria, the pressure Poisson equation is generally employed for the primitive variables formulation. To investigate this incompressibility constraint, assuming the viscosity is constant and rewriting Equation \ref{eqn:incompressible} with  {\Large $\f{\p u_k}{\p x_k}$}$=0$, \normalsize
\be
%\f{\p u_i}{\p t}+ u_j \f{\p u_i}{\p u_j}=-\f{\p p}{\p x_i}+\f{\p \tau_{ij}}{\p x_j}
\f{\p u_i}{\p t}+ \f{\p u_i u_j}{\p x_j}=-\f{1}{\rho}\f{\p p}{\p x_i}+\nu \f{\p^2 u_{i}}{\p x_j \p x_j}+f_i
\ee
Then taking the divergence of the above equation with the constant density assumption,
\be
\f{\p}{\p t}(\f{\p u_i}{\p x_i}) + \f{\p}{\p x_i}(\f{\p u_i u_j}{\p x_j})=-\f{1}{\rho}\f{\p^2 p}{\p x_i \p x_i} + \nu \f{\p^2}{\p x_j \p x_j}(\f{\p u_i}{\p x_i})
\ee
The above equation can be rearranged as:
\be
(\f{\p}{\p t}-\nu \f{\p^2}{\p x_j \p x_j})\f{\p u_i}{\p x_i}= h
\label{eqn:NS-div}
\ee
where $h$ can be viewed as a source term,
\be
h=-\f{\p}{\p x_i}(\f{\p u_i u_j}{\p x_j})-\f{1}{\rho}\f{\p^2 p}{\p x_i \p x_i}
\label{eqn:pressure-poisson-h}
\ee
Equation \ref{eqn:NS-div} means that the increase rate of the velocity divergence, {\Large $\f{\p}{\p t}(\f{\p u_i}{\p x_i})$}, equals the diffusion term, {\Large $\nu \f{\p^2}{\p x_j \p x_j}(\f{\p u_i}{\p x_i})$}, plus the source term, $h$.
To ensure the divergence-free constraint is satisfied everywhere at all time, the initial and boundary conditions of velocities have to be divergence-free; and the source term $h$ is required to be equal to zero. This means that the advection term which caused the non-zero divergence, {\Large $\f{\p}{\p x_i}(\f{\p u_i u_j}{\p x_j})$}, has to be counterbalanced by the pressure gradient. It seems straight forward to find the pressure Poisson equation by forcing $h=0$ in Equation \ref{eqn:pressure-poisson-h}. However, small errors accumulated in the computation of this equation will be integrated in time, eventually resulting in significant contribution to the divergence of velocity. A more effective approach called the projection method\cite{Chorin1980} is used instead.

If the density is not treated as constant, then the divergence of Equation \ref{eqn:incompressible} shows,
\be
\f{\p }{\p t}(\di \+u)+\di (\+u \cdot \n \+u)=\nu \di (\n^2 \+u) - \n (\f{1}{\rho})\cdot \n p - \f{1}{\rho}\n^2 p
\ee
or,
\be
[\f{\p }{\p t}+\nu \n^2](\di \+u) = -\n (\f{1}{\rho})\cdot \n p - \f{1}{\rho}\n^2 p-\di (\+u \cdot \n \+u)
\ee

It can be seen that the incompressibility constraint is satisfied if and only if the right hand side of the equation is zero everywhere at all time:
\be
-\n (\f{1}{\rho})\cdot \n p - \f{1}{\rho}\n^2 p + \di (\+u \cdot \n \+u)=0
\ee
When $\n (\f{1}{\rho})\cdot \n p$ is negligible compared with $\f{1}{\rho}\n^2 p$, it can be approximated as:
\be
\f{1}{\rho}\n^2 p =- \di (\+u \cdot \n \+u)
\label{eqn:pressure-poisson}
\ee

%Halliday , Resnick, Walker, Fundamentals of Physics, 5E,Extended, Wiley 1997
%http://en.wikipedia.org/wiki/Pascal

