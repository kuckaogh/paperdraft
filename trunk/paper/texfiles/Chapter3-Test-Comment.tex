
\begin{comment}
cite:

There are three types of error in modeling: (1) numerical error caused by round off and/or
truncation, (2) physical error attributed to inaccurate parameter estimation, and (3) error
that is traceable to limited or poor-quality data. RSM calibration and validation examples
should identify these three sources of error. Numerical error can be minimized by a
judicious choice of grid resolution and time step and physical error can be minimized by
the proper choice of parameter values, while data-quality error usually can be assessed
only qualitatively. However, the importance of data-quality error cannot be
overemphasized. Full model validation requires explicit separation of error; otherwise,
one could be calibrating numerical errors against physical and/or data-quality errors. The
validation procedure should take into account the following considerations: (1) to the
extent possible, eliminate numerical error; (2) calibrate physical parameters to acceptable
values; and (3) if necessary, assess the quality of measured input data.
The Panel is reassured that the SFWMD will make every effort to distinguish between the
three types of error which arise in mathematical modeling. First, numerical errors should
be minimized; second, physical errors should be investigated, identified, and corrected;
and third, data-quality errors should be acknowledged and, to the extent possible,
resolved. As the SFWMD has adroitly recognized, disregarding this triad results in bad
modeling practice.
The issue of calibrating physical parameters to acceptable values is controversial. One
group of individuals with expertise in this area would argue that the constraints on the
physical parameters should be limited to realistic values. This allows modelers to
determine the parameter values that best fit the observed data. These optimal parameters
can be compared to realistic parameter ranges in order to assess the conceptual validity of
the model. Another group of experts would argue that physical parameter ranges should
not be constrained in order to enforce the conceptual basis of the model. In this case,
extreme and often unrealistic values of the optimal parameters would serve as an
indication that conceptual problems might exist in the model. To accommodate both of
these views, a model could have the option of either specifying acceptable ranges of
physical parameters or not constraining these parameters at all. The modeler would then
interpret the estimated physical parameters accordingly. The Panel recognizes that the
inclusion of tools and techniques to constrain model parameters to acceptable ranges is
currently part of the long-term RSM development strategy, and the current version of the
RSM provides features that are similar to this recommendation.

\end{comment} 